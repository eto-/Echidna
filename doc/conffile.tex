\section{Configuration files and options}
\label{sec:conffile}
The configuration source in Echidna comes from 2 files and from command line, which are processed by the 
configuration\_manager (see \S\ref{sec:parameters}); this section will explain the syntax and the general 
usage of the files and options. 

Parameters leave in namespaces then to identify them is required the namespace name; the choosed syntax is
{\bf namespace\_name.parameter\_name} which is called {\bf Full Qualified Parameter Name} (FQPN).
Since the namespace name is the same as the bx\_named name owning that namespace this syntax
defines the \code{object\_name.parameter\_name} adressing of the parameter.\\
Relative parameter names can be used only inside the bx\_named objects or in the relative initialization
stanzas (see below), elsewhere a parameter is identified by its FQPN.

\subsection{echidna.cfg}
\label{sec:conf_ech_cfg}
It's the main configuration file; it contains the initialization of the program and should never modified 
by users. It contains a list of {\bf stanzas}; empty lines are skipped as well comments starting with the 
character "{\bf \#}".
A stanza is a set of {\bf assegnation lines} preceded and ended by special keywords as in the following examples:
\vskip 2mm
\begin{minipage}{5cm}
\begin{verbatim}
<module bx_laben_decoder>
priority 20
enable 1
discard_out_of_gate_hits 0
disable_lg { }
<end>
\end{verbatim}
\end{minipage}
\begin{minipage}{\textwidth}
start stanza of type {\it module} and name {\it bx\_laben\_decoder}\\
initialize the parameter {\it priority} to {\it 20}\\
initialize the parameter {\it enable} to {\it 1}\\
initialize the parameter {\it discard\_out\_of\_gate\_hits} to {\it 0}\\
initialize the parameter {\it disable\_lg} to an empty vector\\
end the stanza
\end{minipage}
\vskip 2mm

\noindent The first keyword has the form \code{<{\bf stanza\_type} {\bf stanza\_name}>}; stanza\_name is the name 
of the stanza (an arbitrary string from the point of view of the syntax), while stanza\_type is one of \code{named},
\code{module} and \code{config}. Each stanza is ended by the special keyword \code{<end>}.\\
The stanza body is a list of assegnation with the syntax \code{{\bf parameter\_name} {\bf parameter\_value}(s)} 
(one for line) as shown in the previus example; these assegnations assign parameter\_value values to 
the parameter\_name parameters.

The stanza types are:
\begin{itemize}
\item \code{named}: this stanza initializes a parameters namespace whose name is stanza\_name; the role of 
such stanzas is to initialize the default status (parameter values) of bx\_named objects in Echidna.
Each assignement line initialize the specified parameter to the specified value; initializing a parameter
differs from a setting it as explained below and in \sec{custom_param}.

The meaning of the parameters and the accepted types/values fully depend on the target bx\_named objects;
one of the role of these initialization stanzas is to collect the information on the parameters in one file
with their default values (and maybe some comments).

In these stanzas, since the name of the targets bx\_named is well known, the parameter addressing is done only
by the relative parameter name (simply its name).
\item \code{module}: initialize a bx\_base\_module object with name stanza\_name in the same way a named
stanza initialize a bx\_named object. Moreover the specified module is added to the list of objects created by the
module\_factory (\S\ref{sec:modfactory}).
The initialization rules and the paramer addressing rules are the same than in a named stanza.

Some predefined parameters are relevant:
\begin{itemize}
\item \code{enable}: can be 0 or 1 for disabling or enabling the module.
\item \code{priority}: select the priority (integer values) of the module in the framework. The modules with low priority
are executed before modules with high priority.
\end{itemize}
\item \code{config}: create a configuration with name stanza\_name which can be choosed at runtime by Echidna arguments.
Configurations are collection of parameter redefinements whose role is to override the default state of bx\_named 
(and modules) in atomic operations, instead of overriding each parameter manually\footnote{Config stanzas are usefull to
set Echidna in some often used configurations.}

A config stanza body is a list of assegnations where parameter are identified by their FQPN; once a specific configuration
is selected those assignement will be set (not initializated); later reconfiguration can be done by user file or by 
command line.

At list one configuration has to be present, the {\bf default} one (even with empty body); when no other configurations
are specified on command line, this configuration is used.
\end{itemize}

\subsection{user.cfg}
\label{sec:conf_user_cfg}
The role of this file is to specify a set of overrides local to the user. The file consists of
a list of assegnation lines (almost with the same syntax of the config stanza body); parameters
are addressed by their FQPN, empty lines and comments (starting with '\#' character) are skipped.

A small example from my code follows; I normally run without muons, my default file is Run 1612 (with
the specified path) and sometime I happen to disable the \code{bx\_precalib\_laben\_check\_tdc} so that line
is commented.
\begin{verbatim}
bx_event_reader.file_url file:///mnt/scratch/Borex/rawdata/Run001612_01.out.gz
bx_precalib_muon_findpulse.enable 0
bx_precalib_muon_pedestals.enable 0
#bx_precalib_laben_check_tdc.enable 0
bx_muon_decoder.enable 0
\end{verbatim}


\subsection{Echidna inline options}
\label{conf_options}

The main work on options is still to be done, only few options are present actually:
\begin{itemize}
\item \code{-v} increments the Echidna verbosity of one level (\S\ref{sec:messages}); several -v can be
present to increment further the verbosity (actually since the default verbosity value is warn, only
3 -v are usefull).
\item \code{-q} decrement the Echidna verbosity of one level (\S\ref{sec:messages}); several -q can be
present to decrement further the verbosity (actually since the default verbosity value is warn, only
2 -q are usefull). This option is antagonist to -v.
\item \code{-l <file\_path>} specify a logfile with path {\code file\_path}; if file\_path is absent no
logfile is generated.
\item \code{-c file\_path} specifies the path of the user.cfg file (default ./user.cfg)
\item \code{-C file\_path} specifies the path of the echidna.cfg file (default ./echidna.cfg)
\item \code{-p FQPN value} set to value the specified parameter; this assignement has the highest
priority.
\item \code{-f file\_url} set the input file url for the reader, equivalent to "\code{-p bx\_event\_reader.file\_url file\_url}.
\item \code{-o file\_url} set the output file path for the writer, equivalent to "\code{-p bx\_event\_writer.file\_name file\_path}
\item \code{-e max\_number\_of\_events} set the maximum number of processed events.
\end{itemize}

After the options (specified with a '-' character) the first words is the configuration name to be used; this configuration
is checked to be present on the echidna.cfg file and if not an error is issued. Else the specified configuration is
applied; if no configuration is given than the default one is used.
Other fields after the configuration name are meaningless and generate an error.

\subsection{Setting parameter}
\label{sec:conf_param}
The order in with the assignement are applied is relevant and is:
\begin{enumerate}
\item initialization stanzas (named and module)
\item configuration (only the config stanza selected by user)
\item user.cfg
\item command line options specified with -p (or equivalent shortcuts).
\end{enumerate}

The assegnations other than which present in named and module stanzas do not initialize
the parameters; this is not a critical problem. By the way a warning is generated to alert
the user who is setting a uninitializated parameter\footnote{The following warning is issued:\\
\code{\scriptsize
>>> warn: bx\_event\_reader: assigning value to unknown parameter "junk" will initialize it. Check for syntax error
}\\
Which report for an uninitializated parameter \code{junk} and module \code{bx\_event\_reader}.}.
