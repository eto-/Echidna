\section{CVS usage and policy}
\label{sec:cvs}
The Echidna code is handled by CVS a "{\it version control system}" which is able 
to record the history of source files. An exhaustive documentation of CVS is available
using info (accessible by the command \code{info cvs}); a man-page is also available with
a shorter, but more obscure, usage. Reading the info pages is strongly suggested to learn 
the CVS usage, its architecture, its pitfalls.
In this section a certain familiarity with cvs environment is assumed.

The Echidna CVS repository is \code{bxmaster.lngs.infn.it:/home/cvs} which is automatically set
if using the \code{.bashrc} from \code{/home/bashrc} in the LNGS cluster: the Echidna module is 
called \code{Echidna}. Then to download the latest development version the following code 
has to be issued:  \code{cvs checkout Echidna}

\begin{figure}
\begin{center}
\includegraphics[width=0.9\textwidth]{pictures/cvs_branches}
\caption{CVS branches in Echidna}
\label{fig:cvs_branches}
\end{center}
\end{figure}

Echidna is developed using {\bf branches}, a CVS feature which allows package versioning; please
refer to the {\it Branching and merging} CVS info section for an exhaustive documentation on
branches. A branch is a separate line of development; when a file is changed on a branch, those changes
do not appear on the main trunk or other branches. Infinite branches can coexist in a CVS module, but
in Echidna a well defined scheme is present.\\
Since the development is cycle oriented, the branches layout will respect this approach. Figure~\ref{fig:cvs_branches}
shows how the project branches should appear: the main trunk, which is represented as an orizzontal 
solid black line in figure~\ref{fig:cvs_branches}, is called {\bf head branch}; the blue line represent
branches. The development model is the following:
\begin{itemize}
\item free development is done in the head branch during the cycle (according with the cycle objectives).
\item At the end of the free development, when the code is considered somehow stable, a branch is created.
The name of the branch is "{\bf cycle\_n}" where n is the cycle index.
\item The test procedure should validate the branch code. In the branch only bug fixes are allowed; no new code
can be introduced.
\item {\bf The fixes to a code on validation must be committed ONLY on the current cycle branch}, not on the main
trunk. The modifications will be introduced into the head branch during the merging operation\footnote{To control
which version of file is in use on a local copy of Echidna, the command cvs status file, reports the current version
with the followig syntax: "\code{Sticky Tag: cycle\_1 (branch: 1.1.2)}".\\
Here the \code{Stycky Tags} says that the current cvs.tex file version owns to cycle\_1 branch.
}.
\item When the test procedure finish the branch is merged in the main trunk (dashed line). 
\item To avoid conflicts during the merges the head branch is partially frozen; no changes must be made
to the code being validated. Only new modules can be committed. These changes will not become part of the
current branch, but will be on the next branch.
\item Creating branches and merging them is and operation reserved to the CVS manager. 
\end{itemize}
This policy ensures that at least a stable version is always present for analisys; the last cycle branch;
indeed the merging procedure does not destroy the branch but pushes the modification in the main trunk.
This way the cycle branches will be always available for use, while the main trunk is present for development.

To access a branch checkout or update CVS command can be used with the flag \code{-r branch\_name}; since a local 
copy keeps memory of the branch in use when committing, it is important to check which is the branch in use before 
committing, else a commit can violate the policy or even commit on a closed branch.
{\bf Never specify a tag with -r branch\_name on commit}\footnote{Even if CVS accept "\code{cvs ci -r tag file}"
commands the meaning of this syntax does not conforms with the brach engine used in Echidna}. 
Some examples (comments start with a '\#' character):


\begin{verbatim}
~$ cvs co -r cycle_1 Echidna                    #Checkout Echidna with brach cycle_1
~/Echidna$ cvs status doc/cvs.tex               #Control the status of a cvs.tex file
===================================================================
File: cvs.tex           Status: Locally Modified

   Working revision:    1.1.2.1
   Repository revision: 1.1.2.1 /home/cvs/Echidna/doc/cvs.tex,v
   Sticky Tag:          cycle_1 (branch: 1.1.2) #Fine it is in cycle_1 branch
   Sticky Date:         (none)
   Sticky Options:      (none)

~/Echidna$ cvs ci doc/cvs.tex                   #Commit cvs.tex in the curren branch (cycle_1)
~/Echidna$ cvs update -r HEAD .                 #Update current directory to the HEAD branch.
\end{verbatim}

By the way using this policy is simpler than describing it: users should always use the latest stable version (which
is the latest cycle branch), as well testers. Developers should work on the head branch, except when doing bug-fixes
to a code under validation.

\subsection{.cvsrc}
It is somehow useful to create a file with that name containing some configuration for CVS, the file must reside in the 
users'home directory; the suggested configuration is:
\begin{verbatim}
cvs -z6
diff -u
update -Pd
checkout -P
\end{verbatim}

The first line enables compression of data transfer over network.
The -u switch tells diff to use the {\it unified diff format} useful for patches. The flags -N can be even useful when dealing with new files.
The -P flags tells to update and checkout to prune empty directories, while -d tells to update to create all subdirs present in the repository.
See the man page for further explanations.
