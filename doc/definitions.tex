% Echidna documentation 
% this file is for personal aliasis
 
%%%%% davide.dangelo %%%%%%%%%%%%%%%%%%%%%%
%---------------------------- shortcuts ----------------------

% Definitions for lists, equations, ...
\newcommand{\bit}{\begin{itemize}}
\newcommand{\eit}{\end{itemize}}
\newcommand{\ben}{\begin{enumerate}}
\newcommand{\een}{\end{enumerate}}
\newcommand{\bde}{\begin{description}}
\newcommand{\ede}{\end{description}}
\newcommand{\bqu}{\begin{quote}}
\newcommand{\equ}{\end{quote}}
\newcommand{\bmp}[1]{\begin{minipage}[b]{#1\textwidth}\centering}
\newcommand{\emp}{\end{minipage}}

\newcommand{\beq}{\begin{equation}}
\newcommand{\eeq}{\end{equation}}
\newcommand{\bea}{\begin{eqnarray}}
\newcommand{\eea}{\end{eqnarray}}
\newcommand{\bdm}{\begin{displaymath}}
\newcommand{\edm}{\end{displaymath}}

%---------------------------- graphics ----------------------------------

% Captions
\renewcommand{\figurename}{Fig.}
\renewcommand{\tablename}{Tab.}

\newcommand{\mycapfont}{\linespread{0.9} \small \sl} % used 2 times: \captionfont and subfigure captions

\renewcommand{\captionfont}{\mycapfont} % affect label and text
\renewcommand{\captionlabelfont}{\rm} % affects label only, used to undo the above. 

\makeatletter
  \newcommand{\capfig}[2]{\def\@captype{figure}\caption{\label{fig:#1} #2}}
  \newcommand{\captab}[2]{\def\@captype{table}\caption{\label{tab:#1} #2}}
\makeatother

% Definitions for figures and tables.
\newcommand{\bfig}{\begin{figure} \centering}
\newcommand{\efig}{\end{figure}}
\newcommand{\btab}{\begin{table} \centering}
\newcommand{\etab}{\end{table}}
\newcommand{\bff}[2]{\begin{floatingfigure}[#1]{#2\textwidth}}
\newcommand{\eff}{\end{floatingfigure}}
\newcommand{\bft}{\begin{floatingfigure}}
\newcommand{\eft}{\end{floatingfigure}}
\newcommand{\btr}[1]{\begin{tabular}{#1}}
\newcommand{\etr}{\end{tabular}}

% Internal or special use only
\newcommand{\getfigw}[2]{\includegraphics[width=#2\textwidth]{#1}}
\newcommand{\getfigh}[2]{\includegraphics[height=#2\textheight]{#1}}

\makeatletter
\renewcommand{\@thesubfigure}{\figurename~\thefigure\space\thesubfigure:}
\makeatother

\newcommand{\subfig}[3]{\subfigure[\mycapfont #3]{\label{fig:#1}\getfigw{#1.eps}{#2}}}

% Single figure
\newcommand{\simfig}[3]{\bfig \getfigw{#1.eps}{#2}  \capfig{#1}{#3} \efig}

% Double figure, single caption
\newcommand{\doublefigs}[6]{\bfig \getfigw{#1.eps}{#2} \hfill \getfigw{#3.eps}{#4}  \capfig{#5}{#6} \efig}

% Double figure, double caption, single numbering
%\newcommand{\doublefigd}[6]{\bfig \subfig{#1}{#2}{#3} \hfill \subfig{#4}{#5}{#6} \addtocounter{figure}{+1} \efig}

% Double figure, double caption, double numbering
%\newcommand{\doublefigp}[6]{\bfig \bmp{#2} \getfigw{#1.eps}{1} \capfig{#1}{#3} \emp \hfill \bmp{#5} \getfigw{#4.eps}{1} \capfig{#4}{#6} \emp \efig}

% Triple figure, triple caption, triple numbering
%\newcommand{\triplefigp}[9]{\bfig \bmp{#2} \getfigw{#1.eps}{1} \capfig{#1}{#3} \emp \hfill \bmp{#5} \getfigw{#4.eps}{1} \capfig{#4}{#6} \emp  \hfill \bmp{#8} \getfigw{#7.eps}{1} \capfig{#7}{#9} \emp \efig}


\newcommand{\fig}[1]{fig.~\ref{fig:#1}}
\newcommand{\tab}[1]{tab.~\ref{tab:#1}}
\newcommand{\eq}[1]{eq.~\ref{eq:#1}}
\renewcommand{\sec}[1]{sec.~\ref{sec:#1}}
\newcommand{\ch}[1]{chap.~\ref{ch:#1}}

% Chapter: Borexino

\newcommand{\iv}{\emph{inner vessel}}
\newcommand{\ov}{\emph{outer vessel}}
\newcommand{\al}{\ensuremath{\alpha}}
\newcommand{\be}{\ensuremath{\beta}}
\newcommand{\albe}{\ensuremath{\alpha/\beta}}
\newcommand{\ga}{\ensuremath{\gamma}}
\newcommand{\si}{\ensuremath{\sigma}}
\newcommand{\pg}{\ensuremath{\pi}}
\newcommand{\ura}{\ensuremath{^{238}U}}
\newcommand{\tho}{\ensuremath{^{232}Th}}

% Chapter: DAQ

\newcommand{\code}[1]{\texttt{#1}}
\newcommand{\qcode}[1]{\bqu \texttt{#1} \equ}

%%%%%%%%%%%%%%%%%%%%%%%%%%%%%%%%%%%%%%%%%%%%
