\section{Event Structure}
\label{sec:event_structure}

This section describes only the event structure used internally by echidna.
For the root event structure see \sec{root_event} and \sec{root_event_for_users}.

In this section two concepts are introduced:
\begin{description}
\item{\emph{detector segment}.}
This refers to different components that reflect the Borexino detector structure: ``trigger'', ``laben'', ``muon'', ``fadc'' (flash adc).
Two more segments, ``mctruth'' (Monte Carlo truth) and ``pid'' (particle ID) are also present that do not correspond to a physical DAQ sub-system.  
\item{\emph{reconstruction stage} and \emph{reconstruction step}.}
These refer to the iter the event is undergoing in the program. 
A step brings the event into its next stage.
Reconstruction steps can be: deocoding, clustering, splitting, reconstruction, pulse shape discrimination, ...
Reconstruction steps can be: raw, decoded, clustered, split, reconstructed, discriminated, ...
A module is normally in charge of performing a reconstruction stage.
Different detector segmentss may have different reconstruction steps/stages.
\end{description}

At the moment of writing the structure is not yet frozen; 
particularly it is under discussion the possibility of adding a pseudo-segment called ``global\_event'' 
to collect high level information not related to a particular detector segment.

\bfig
\includegraphics[angle=270, width=\textwidth]{event_structure.eps}
\capfig{event_structure}{Echidna internal event structure.}
\efig

\subsection{Event Classes}

The event structure is depicted in figure \ref{fig:event_structure}.
The most general class in this context is the \code{bx\_echidna\_event}.
This class hosts the event header information and provides relative getters.
Examples of these are the event number, the run number, the crates enabled at run time, ...

Then this class holds 6 objects, one for each detector segment:
\ben
\item \code{bx\_trigger\_event}
\item \code{bx\_laben\_event}
\item \code{bx\_muon\_event}
\item \code{bx\_fadc\_event}
\item \code{bx\_pid\_event}
\item \code{bx\_mctruth\_event}
\een

Every one of the above class inherits from different classes, each one containing information relative to a different reconstruction stage appropriate to that detector segment.
For example the bx\_laben\_event inherits from:
\begin{itemize}
\item \code{bx\_laben\_raw\_event}
\item \code{bx\_laben\_decoded\_event}
\item \code{bx\_laben\_clustered\_event}
\item \code{bx\_laben\_rec\_event}
\end{itemize}

\subsection{Hit Classes}
 
If appropriate a hit class is defined for each reconstruction stage.
In this case the event class holds a vector of these hits.
For example class \code{bx\_laben\_decoded\_event} holds a \code{std::vector<bx\_laben\_decoded\_hits>}, while the getter:

'\code{const bx\_laben\_decoded\_hit\& bx\_laben\_decoded\_event::get\_hit(int i);}' 

returns the hit \#i.

Since the module performing a given reconstruction step may decide to discard hits according to its own criterion, vector of hits in successive reconstruction stages may be disaligned.
For this reason the index is not a good key to retrieve correlated information. 
Instead every hit class has a pointer to its correlated hit in the previous stage.

In addition decoded hit classes have pointers to a \code{db\_channel} object containing database information on the relative channel.
These pointers are assigned by the decoder module the first one in the main loop, therefore they are not available during precalibration.

\subsection{Clusters, Fragments, Windows}

Starting with clustered stage, laben hits are organized in clusters (up to 3 per event) according to the result of clustering modules.
Hits not belonging to any cluster are discarded in this step. 
Clusters refer to physical events within the Borexino trigger gate that are not piled up, i.e. whose time separation is not below 150ns.
Depending on the performance of the splitting module (in development at the time of writing) clusters may be traduced in fragments at a later stage, where also piled up physical events are separeted.

Fadc data are ``naturally'' (i.e. by the daq) organized in windows. 
Windows contain channel objects in early reconstruction stages. 
Clustering for this detector segment limits windows to the 3 most significant ones, hopefully aligned with laben clusters. 
