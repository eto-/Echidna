\section{HOWTO run Echidna}
\label{sec:run}

System requirements: Operating System, CPU, memory, compilers, libraries and packets installed, etc. are defined in Release Notes for every cycle. 

If you are running on the Borexino cluster in Gran Sasso (e.g. on \code{bxmaster}) these requirements are automatically met as the system administrator takes care of it.
However, as mentioned in the cluster tutorial (http://bxweb.lngs.infn.it/echidna/cluster\_tutorial.html), 
this requires that the \code{.bash\_rc} and \code{.bash\_profile} configuration files in your home area are including (sourcing) the file \code{/home/environmental\_setup}.


\subsection{Getting the code}
\label{sec:run_get}

The Echidna code is stored in Borexino official cvs repository in Gran Sasso.

In the Borexino cluster cvs is already set up for you.
Otherwise you simply need to set the following environmental variable:
\qcode{CVSROOT = bxmaster.lngs.infn.it:/home/cvs/}

Then you download the code with the following standard cvs command:
\qcode{cvs co -r cycle\_2 Echidna}
where ``\code{cycle\_2}'' is the name of the branch you want to get.
If you don't specify it (i.e. you omit the \code{-r} option), you get the latest branch.
This may eventually be an unstable developing version, not the right one for testing.


\subsection{Compilation}
\label{sec:run_make}

A Makefile is provided, simply do:
\qcode{make}

This generates three files:
\bde
\item[\code{echidna}]. The executable you will run.
\item[\code{libechinda.so}]. Echidna internal library. 
  Nothing special about it, just leave it in peace.
\item[\code{root/rootechidna.so}]. Echidna ROOT library. 
  You will need to load this into ROOT (after echidna exedcution) when you want to look at the ROOT file echidna has produced for you (\sec{root_event_for_users}). 
\ede


\subsection{Setting up the Database connection}
\label{sec:run_db}

If you are working on Borexino Cluster there is nothing you have to do, skip to next section.

If not, you will need to setup the IP address of the DB server, before you can run.

This can be done adding the following line to the file \code{user.cfg} (\sec{custom_user_cfg}) in the main Echidna directory:
\qcode{bx\_dbi.SERVER1 bxdb.lngs.infn.it}

In addition, the IP address \emph{from}  which you connect to DB must be in the list of authorized addresses of the Postrgres server in Gran Sasso.
If you are working on a machine of an institute of the Borexino collaboration this may be already the case.
If not or if you are running at home, you need to send an email to the cluster administrator with the IP address you want to be authorized.

If you want to use a different DB server (mirrors in other institutions) use that address and contact its system administrator for authorization.

If finally you want to install a local copy of the database, you find instructions in\cite{???}.
Then you will need:
\qcode{bx\_dbi.SERVER1 127.0.0.1}
in your \code{user.cfg} file


\subsection{Running Echidna}
\label{sec:run_run}

You can run echidna calling directly the executable (no launching script is used.):
\qcode{./echidna -f run://XXXX}
where XXXX is the number (not the filename!) of the run you want to process .

The rawdata file will be automatically retrieved through TCP/IP from the official Borexino storage in Gran Sasso.

If you need instead to process a local rawdata file, you can do it with:
\qcode{./echidna -f filename}

