\chapter{Introduction}
\label{ch:intro}

This is the full documentation of the Echidna project, the offline data processing software of the Borexino experiment.
It is meant to be exaustive.
Testers and plain users of the program may bettere jump to \sec{run}.
Module developers may better read \sec{modules}.

The Echidna is a small ant-eater that lives in Australia but Echidna is also the name of a mythological figure of ancient Greece, half woman and half snake she was the daugther of \ldots

The project started in March 2004 and in about six months a working infrastructure was built.
At the moment of writing, about 30 mdoules are operative covering most of the tasks listed above, many of which are currently being tested.

\section{Physics tasks}
\label{sec:intro_tasks}

The main physics goals of the offline code are:
\ben
\item Predecoding (a.k.a. precalibration). Electronics speficic parameters computed for a given run or profile.
\item Decodification of the raw information.
  \ben
  \item Decode GPS time for absolute time reference.
  \item Laben hit time and charge must be inferred from ADC bins using precalibration and calibration data.
  \item The same is true for outer muon hits from TDC bins.
  \item For FADC the parallel operations are pedestal subtraction and the bulding of a digital sum of channels.
    \een
\item Clustering of hits into physical events and precise evealution of event start time. FADC must select the proper windows and provide redundant information.
\item Vertex position reconstruction of the event.
\item Splitting of overlapped events within a cluster.
\item Reconstruction of the event energy.
\item Computation of PID tags:
  \ben
  \item \albe discrimination.
  \item Muon tagging.
  \item Sphericity of the event and other geometrical parameters.
  \een
\item Muon track reconstructions.
\item Calibration tasks:
  \ben 
  \item Single channel calibration: timing and charge for ID and OD.
  \item Dark rate of PMTs.
  \item Scintillator and water quality evaluation.
  \item Electronics status evaluation.
  \item Calibration system auto-monitor.
    \een
\een
and the list is of course not exaustive, nor it can accounts for new ideas being delivered every month as the work proceed.


\section{Design principles}
\label{sec:intro_design}

The guiding principles of the design are:
\bde
\item[Modulary]. 
  The physics algorythms are organized into indipendent modules with a fixed intarface and run by a general engine. 
  The main engine and large set of ancillary routines constitute the \emph{infrastructure} of the program.

\item[Data protection]. 
  Every module is assigned an area with an event class of which it is responsible and it is not allowed (i.e. compilation fails) to write elsewhere. 
  In this way it is immediately possible to track a bug or a mis-computed parameter to its very source.

\item[Use of ROOT]. 
  The collaboration agreed that the official analysis platform in use in Borexino is ROOT\cite{???} . Echidna therefore produces its output as a standard ROOT file. 

\item[Use of Database.] 
  The program is able to access all sort of required information (mapping, geometry, calibration, fluid handling) from the Borexino database.
  A few tables (e.g. precalibration) are also written by echidna.

\item[Single language: C++]. 
  A robust offline program had to be coded using one programming language only and the most natural choice was C++ as it is the language used by ROOT and by its interpreter.
  Moreover the Postgres database to which the program interfaces provides API for C++ and not for Fortran.

\item[\emph{Object-Oriented} (OO) programming]. 
  This design phylosphy comes pretty natural in C++ and constitutes an advantage in enforcing modularity and data protection. 
  However there are different degrees to which it can be pushed and it was decided that the accessibility of the code to a large group of developers was a higher requirement then a truly orthodox OO programming style. 

\item[Network oriented.] 
  Database and rawdata files are accessed over TCP/IP at run time from the DB sever and the disk storage in Gran Sasso\footnote{
    In the future it is foreseen building a mirror in the Lion (France) in2p3 computing facility.} respectively.
  Local copies are needed only if running of an isolated machine.
\ede


\section{Organization}
\label{sec:intro_cycles}

\simfig{cvs_branches}{0.9}{The time evolution of Echidna working cycles.}

The working group is composed of about twenty collaborators, equally distributed among \emph{developers} and \emph{testers}.

Testers have a fundamental role: they have to understanding the module algorythm (from documentaion, without readinbg the code), 
produce a set of relevant simulated and/or real data and perform a list of tests on the module behaviour, finally filling a written report to be archived. 

Every module is assigned a maintaner (usually the author) and a tester.

The interaction among developers working parallely is handled with CVS\footnote{\emph{Concurrent Version System}, a standard tool for this goal\cite{???}.} 
and the code is stored in the official Borexino CVS repository in Gran Sasso.

Under the guide of a project manager, the work is organized in production cycles lasting about 2 months.
Each cycle includes:
\bde
\item[Meeting]. Every cycle starts with a meeting where the \emph{Request-For-Changes} (RFCs) proposed by any co-worker are discussed, approved and scheduled.
\item[Development]. Then a development period of about 6 weeks follows, during which documentation of the code produced must be also supplied and/or updated.
\item[Branch spawning]. When all scheduled tasks have been accomplished the project manager decides to ``freeze'' the code in a parallel CVS branch.
\item[Tests]. A this point testing and debugging occurs on the parallel branch (about 2 weeks), while the main branch remains avalable for further development.
\item[Merging]. At the following cycle meeting testers report on the past branch and if everything is in order the two branches are merged back into one before the next development period starts. 
\ede

\bfig
\includegraphics[angle=90, width=\textwidth]{ech_scheme.eps}
\capfig{scheme}{Echinda general scheme.}
\efig
