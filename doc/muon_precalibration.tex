\section{Precalibration: Muon}
\label{sec:precalib_muon}

The outer detector requires two precalibration modules.
They must run in order, in two successive precalibration cycles.

\subsection{bx\_precalib\_muon\_findpulse}

This module simply finds the time\footnote{All times used in precalibration are raw times, i.e. they are unsigned long values expressed in TDC ticks (1.0416ns).} 
of the precalibration pulse in the TDC time window. 
The information in used by the \code{bx\_precalib\_muon\_pedestals} module.

It uses a single private \code{std::vector<unsigned long>} as a time histogram for all channels together.
In the \code{doit()} method the vector is filled with lead times of ordinary channels.
In the \code{end()} method custom algorythms from algorythms.hh are used to make compuations.
Mean value is written to appropriate \code{db\_run} variable.

\code{b\_has\_data} is used to prevent unnecessary computation.

\subsection{bx\_precalib\_muon\_pedestals}

This module computes QTC pedestals for every outer muon channel.
The information is used for charge computation by \code{bx\_muon\_decoder}

It uses a private \code{std::vector<unsigned long>} as a histogram for each channel.
In the \code{doit()} method the vectors are filled with relative times differenses (lead-trail).

3 filters are applied in the following order:
\begin{enumerate}
\item Channel type. Only ordinary channels are processed.
\item Pulse time. Only pulses falling within a tolerance from the preclibration time are accepted.
Precalib pulse time comes from previous precalibration and is retrieved through \code{db\_run} object.
Tolerance is set through user parameter \code{precalib\_pulse\_tolerance}.
\item Pulse width. Only pulse falling between min and max values are accepted. 
A Warning is generated for pulses falling outside.
Min and max values are set through user parameters (\code{min\_pedestal\_width} and \code{max\_pedestal\_width} respectively).
\end{enumerate}

In the \code{end()} method custom algorythms from algorythms.hh are used to make computations.
Mean value is written to appropriate \code{db\_run} variable for every ordinary channel.

\code{b\_has\_data} is used to prevent unnecessary computation.
