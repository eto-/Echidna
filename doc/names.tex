\section{Name conventions}
\label{sec:names}

Echidna enforces the use of name conventions for the sake of omogeneity of the code as a whole.
Any Echidna developer is required to follow the guidelines expressed here.


\subsection{Capitals and word separation}
\begin{itemize}
\item 
  Capital letters are never allowed anywhere in the code. 
\item 
  Word separation in symbols is done through '\_' (underscore) character.
\end{itemize}

The only exception to both these rules are the ROOT classes (i.e., those derived from \code{TObject}).
For this ROOT names convention is followed, i.e. the First letter is capitalized and no separator is used among words. 


\subsection{Variables and methods}
Private class members must start with:
\begin{itemize}
\item 
  'i1\_', 'i2\_, 'i4\_' for char, short, int/long respectively.
\item 
  'u1\_', 'u2\_, 'u4\_' for unsigned char, unsigned short, unsigned int/long respectively.
\item 
  'f4\_', 'f8\_' for float and double respctively.
\item 
  's\_' for a string or a C-like char*. 
\item 
  'b\_' for a bool.
\item
  'm\_' for helper method.
\end{itemize}

the above conventions do not apply to ROOT classes.

[SG]etters must match the variable name they want to [sg]et, preceeded by a '[sg]et\_'.


\subsection{Modules}

Most classes and surely all modules must have a name starting with 'bx\_'.
A module is a \emph{device} performing an operation, it is not the operation itself. So, for example, a module performing the splitting will be called \code{bx\_splitter} rather then \code{bx\_splitting}.
The module name must not include the word module.


\subsection{Event references}

The \code{doit()} method of every module recieves a pointer to a \code{bx\_reco\_event}. This should be named simply 'e'.
References to subevents from/to which the module reads/writes should be named 'er'/'ew'.


\subsection{Header comment}
Both '.cc' and '.hh' files should start with a large comment like the following:

\noindent\code{/\** BOREXINO Reconstruction program\\
 \**\\
 \** Author    : Name Surname < email address >\\
 \** Maintainer: Name Surname < email address >\\
 \**\\
 \** \$Id\$\\
 \**\\
 \** This class is ...\\
 \** ...\\
 \** blah blah blah\\
 \** ...\\
 \**\\
\**/\\}

The compuond \code{'\$Id\$'} is expanded by cvs after the first commit and will hold useful informations.

The descriptive comment on the class idea should go into the '.hh' file, the '.cc' should hold a comment like:

\code{* Implementation for bx\_thisandthat.hh}

plus any technical comment upon implementation issue.
In other words, the class user should not need to look at the '.cc' file.

Optionally the file can be ended by a comment like the following:

\noindent\code{/\**\\
 \** \$Log\$
\**/\\}


which cvs will expand with the list of all commit logs.
This can be useful or cumbersome, so its use is left to the file author.


\subsection{Files}

File names must match the name of the class they define/implement.
If more then one class is defined/implemented in a file, the most used one should be used for the file name.


\subsection{Indentation}

Indentation should be done with 2 spaces. 
A number of other suggestions could be specified upon where to put spaces and newlines, 
but the easiest thing, for the developer that wants to stick to general conventions, is to look at existing code.
