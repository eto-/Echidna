\section{Parameters handling}
\label{sec:parameters}
The known problem of passing informations between unrelated classes in C++ is resolved
in Echidna using a software bus technology which will be explained in this section.

A software bus is an engine to pass informations between objects, called {\it clients}:
in Echidna the class called {\bf bx\_named} is the interface to this bus; every class 
inheriting from bx\_named acquires the access to the bus with the intelligence to read 
parameters from it.

Considering the bus standard topology a {\it master} needs some kind of addressing algoritm to send
the informations only to the relevant objects: in Echidna the address of the client is 
a string, the {\it object name} (from wich bx\_named).
Every bus client is a bx\_named progeny identified in the program by a 
{\bf name}; several istances or implementations with the same name are undistinguished 
from the bus and thus all them have the same parameter set.\\
The bus itself is implemented by the {\bf parameter\_broker} class, while the bus master
job is done by the {\bf configuration\_manager} class; the objects traveling the bus
are {\bf vdt} istances (see \S\ref{sec:vdt}).

This technology is used in Echidna for parameters handling: a parameter is a configuration 
data not hardcoded in the program sources. Parameters are normally initialized in the main
configuration file {\bf echidna.cfg}, set by the current configuration, overrided by the
user configuration file {\bf user.cfg} and by the command line arguments. Parameters differ
from constants which can not be changed without modifying the sources. On the countrary
parameters should not used between modules to pass calculation data (as opposed to 
configuration data): for example the database interface is designed to pass data from 
the precalibrations to the decoding modules.\\
Parameters are identified by their name and organized in {\bf namespaces}: a name has
to be unique in a namespace but a different parameter with the same name can be contained
by a different namespace. The namespace is associated to the bx\_named objects using the
bx\_named names.

\subsection{bx\_named}
The main property of a bx\_named istance is having a name; while a bx\_named object is
useless, since it deal only with parameters, it is an important ancestor for many classes
which parameters.

A name is required to construct a bx\_named object and every class inheriting from bx\_named
must satisfy this requirement in the constructor; the 
\mbox{\code{const string\& bx\_named::get\_name () const}} method can be used
later to retrieve the object name.

The main property of a bx\_named is the access to parameters using the parameter name; 
three methods are present:
\begin{itemize}
\item \mbox{\code{const vdt\& bx\_named::get\_parameter (const std::string\& name) const throw runtime\_error}}:
get the parameter with name \code{name}; the search is limited in the namespace associated to the current
bx\_named. If the search fails an exception (\code{runtime\_error}) is generated. The method is \code{const} and
the return value can not be modified.
\item \mbox{\code{bool bx\_named::check\_parameter (const std::string\& name) const}} is used to check if
the required parameter is present allowing to avoid exception generation if the parameter is absent.
\item \mbox{\code{void bx\_named::set\_parameter (const std::string\& name, const std::string\& value)}} and\\
\mbox{\code{virtual void bx\_named::set\_parameter (const std::string\& name, const vdt\& value)}} are used to set a 
parameter; if the parameter is not already present a warning is issued. The second method is virtual since the 
childs can redefine it to have cached parameters\footnote{Since the access to the parameter requires two
string searches in the parameter\_broker maps, parameter often accessed could be slow and may be better to cache the 
value locally; one example is the bx\_base\_module (\S\ref{sec:modules}) which caches the \code{enable} parameter. 
Redefining that method is not hard: it is important to call \mbox{\code{bx\_named::set\_parameter ()}} method in the 
redefined version to allow the propagation of the parameters in the software bus. 

There is one trick: problems can arise when there are several istances of the same object (with the same name); 
this way calling \mbox{\code{obj1.set\_parameter ()}} affect the cached parameter, the parameter\_broker version, 
but not the version of a obj2 with the same name of obj1. A solution is the use of static cached values even if it can
not be applied always.}
\end{itemize}
For the access to parameters the parameter\_broker is required and thus the bx\_named constructor will generate
an exception if when it's called the broker is not initialized (see below for details on parameter\_broker initializations).

One last method is present: \mbox{\code{bx\_message\& bx\_named::get\_message (bx\_message::message\_level level)}} which
return a bx\_message reference already initialized with the object name and the verbosity levels specified in the parameters.
\subsection{parameter\_broker}
When the configuration files are read no bx\_named object exists yet, so propagating the parameters to the
peers would be impossile; for this reason a parameter broker is present in Echidna: it's role is to receive
the parameters anytime and to distribute them upon request. The parameters are organized in namespaces and 
the relationship between parameter namespaces and bx\_named is the name; each namespace has a name and 
when a client want to fetch a parameter the search is done only in the namespace with same name than the bx\_named
client. This is even the reason for which 2 clients with the same name share the same parameter set.

The parameter namespaces are implemented using a \code{typedef map<string, vdt> parameter\_namespace} and internally
the parameter\_broker keeps the list of namespaces using a \code{map<string, parameter\_namespace>}; no 
access control rule is present in the broker, the setting call only require to know the address (name) of the client.
To avoid acl but to remain in a safe condition the access to the parameter\_broker is allowed only to
bx\_named objects which internals grant runtime access only to the parameter of the target object (the object from
which the set method is called \code{onj.set()}).\\
The parameter\_broker role would suggest a global object, like a singleton, but to keep it available only to bx\_named
clients the pointer to the broker is a static private member of the bx\_named class; this way, once initialized, it will 
be available only to the the bx\_named internals. The initialization is done, by the configuration\_manager, in the 
\mbox{\code{bx\_options::initialize\_echidna ()}} method.

\subsection{configuration\_manager}
Up to here no code is present to read the configuration files, parse them and send the output to the parameter\_broker;
this is the target of the configuration\_manager.

The configuration source in Echidna comes from 2 files and from command line; the explanation of the meaning of these
sources as their format is left to \S\ref{sec:conffile}, here only some important informations are reported:
\begin{enumerate}
\item {\bf echidna.cfg} is the main configuration file organized as a sequence of {\bf stanzas}; each stanza contains 
only the assignement of parameters for bx\_named and can be one of the following type:
\begin{itemize}
\item \code{named}: initialize the contained parameters for the bx\_named with the same name as the stanza
\item \code{module}: as before but the bx\_named is added to the module list
\item \code{config}: a user selectable configuration, containing values to be set for the contained parameters: parameters
are {\bf fully qualified} (their name contain the bx\_named name with the syntax \code{bx\_named\_name.parameter\_name}). 
When a configuration is selected the setting are applied after the initialization of the named/module stanzas. 
The configurations override the default setting for bx\_named parameters.
\end{itemize}
\item {\bf user.cfg} is the user configuration file; it contains only fully qualified parameter assignements.
\item {\bf command line} generates fully qualified parameter assignements with the highest priority.
\end{enumerate}

The configuration\_manger is created in the \mbox{\code{bx\_options::initialize\_echidna ()}} method, which send to it 
the configuration file names and the command line parameters; then the configuration\_manager is ready to upload the 
the parameter\_broker with the parameter with the following order:
\begin{enumerate}
\item initialization stanzas (named and module)
\item configuration stanza for the current configuration (specified on command line)
\item user configuration parameters
\item command line parameters.
\end{enumerate}
The list of module names is also available to be sent to the bx\_module\_factory for the creation of 
the modules, but before this the parameter\_broker pointer has to be stored in the bx\_named static pointer else
no bx\_named object could be created (and thus modules)\footnote{An interesting hack is that the configuration\_manager
inherits from the parameter\_broker; this way the configuration\_manager gets persistent as well as the broker and
even if it is not requested it helps in debugging.}.
