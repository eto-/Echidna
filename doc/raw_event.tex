\section{Raw Event constructors}

The file reader module invokes in cascade the raw event constructors passing them only a pointer to a memory slab where the event has been read from file.
So possibly a raw event constructor can host intellicence necessary to build the raw event.

Generally the first operation every one of the following constructor does, is casting the pointer it recieves to its own \code{struct event\_disk\_format.h}\footnote{see file \code{bx\_event\_disk\_format.h} for details}, so that data words are available in a more user friendly shape.

All constructors return if the first word of their event segment (size) is zero.


\subsection{Trigger}
Words of \code{trigger\_disk\_format} structure are simply copied to analogous private variables, with one exception.
Just before the GPS data, a long word is present, composed by the TAB sum value (16 bits) and by the so called Trigger Word, which is in turn composed by the trgtype (LSB) and by a bitfield with active BTB inputs (MSB). Due to network/host byte ordering conversion the order of these subwords is reversed and the unpacking in this constructor takes care of it.
We find in order:
\begin{enumerate}
\item trg type (8 bits)
\item BTB inputs (8 bits)
\item TAB sum (16 bits)
\end{enumerate}

The BTB input subword is copied into a bitfield to enhace accessibility.


\subsection{Laben}

Simple copy into specular variables, both for the event and for hit class.


\subsection{Muon}

This constructor performs the job of coupling edges into raw hits, a relatively complicated task.

A loop is done over edges, with a step of 2.
Edges reporting \emph{event number}\footnote{these edges are present only for DAQ reasons} or flagged as \emph{invalid} are silently discarded.

A local TDC chip counter is kept and incremented whenever an edge, flagged as \emph{last}, is found; 
the TDC data do not contain information on the board number, so this is evaluted through this chip counter.

A method named \code{m\_check\_validity()} is in charge of checking if the two edges could belong to the same QTC pulse.
It performs 2 checks:
\begin{enumerate}
\item Slopes of the two edges must be different and edges must belong to the same TDC channel. 
If this is not the case then probably an edge was lost. The first edge is then spurious and is being discarded with a debug level message;
the second edge instead could be a valid first edge for the following hit and will be reprocessed. 
Control is returned to the loop which continues with appropriate counter correction.
\item The time ordering of the two edges must be correct, i.e. the first edge time must be higher then the second one
\footnote{Remember that in common stop mode the TDC yields times which go backward from trigger instant.}.
If this is not the case, both edges are simply discarded with a debug level message. 
Control is returned to the loop which simply continues.
This patological behaviour appears usually when an LED is firing at high intensity, resulting in a high after pulse rate. 
\end{enumerate}
If both checks are passed, a raw hit is constructed and pushed into the vector.

The raw hit constructor simply initializes the two edges and the the computed board number.

A final check on TDC chips complains if the total number doesn't meet expectations (8 chips).


\subsection{Fadc}

This contructor copies event level words into analogous variables, then loops over windows to construct them.
A debug level message is issued if no windows are present.

The raw window constructor copies window level words into analogous variables.
If window size is 0, execution is aborted with critic level message.
The number of channels is deduced from data size and window size and channels are constructed in a loop.

The fadc channel constructor simply copies the samples into a private vector.