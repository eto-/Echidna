\section{Root Event}
\label{sec:root_event}

This section describes the \emph{implementation} of the ROOT event in echidna.
Users of the Echidna ROOT file who look for the structure and meaning of variables should intead see \sec{root_event_for_users}. 

\subsection{The Structure}
\label{sec:root_struct}

The event that is being written to the ROOT file is made with a different, though similar, class system 
with respect to the one used internally by Echidna (see \ref{sec:event_structure}).
This class system is depicted in fig.\ref{fig:root_event}.

All classes inherit from \code{TObject} ROOT class and are being provided a dictionary through rootcint utility.
The \code{bx\_writer\_module} (see \ref{sec:writer}) copies each internal event into the same BxEvent class.
The most notable difference here is that, since the event is written in a single step,
the different subobjects do not need to inherit from classes referring to different reconstruction levels.

The structure is straightforward. The main class \code{BxEvent} holds a subobject for every detector segment:
\begin{enumerate}
\item \code{BxTrigger} Info related to the trigger record
\item \code{BxLaben} Info from ID electronics
\item \code{BxMuon} Info from OD electronics
\item \code{BxFadc} Info from FADC electronics (disabled).
\item \code{BxNeutron} Info from Neutron detection system.
\item \code{BxMcTruth} Info from MonteCarlo simulation chain (empty for real data).
\item \code{BxTrackFitted} Muon tracking info fitted from both ID and OD data (empty for non-muons).
\item a vector of \code{BxPhysTags} objects. Written in DSTs by the bxfilter package. $\#0$ for the event $\#1,2\ldots$ for individual clusters.
\end{enumerate}

Some sub-objects, hold the lists of hits and/or clusters\footnote{or fragments or windows.} directly.
Containers for these lists are \code{std::vector<>}.
Nesting vectors putting hits lists inside culster objects currently prevents the benefit of ROOT splitting and is therefore not implemented.
For reconstruction stages where both hits and clusters are present an index variable in the hit class allows the user to know the corresponding cluster. 
Filling containers is fully custumizable by the user with \code{bx\_writer} parameters (see \ref{sec:writer}).

Every event class has:
\begin{enumerate}
\item A default constructor.

This is required by ROOT file I/O. 

\item A working constructor.

This is the only one called explicitly in Echidna. 
It recieves a global flag structure (\code{bx\_write\_opts}) with data lists which are enabled by the user 
and saves into private variables only the flags concerning itself.

\item An assigment operator.

It recieves a refernce to the correspondent internal event object and makes the copy.
\end{enumerate}

The working constructor and the assigment operator are filtered out through conditional compilation when rootcint runs 
and also when the shared object for ROOT is created.
This eliminates the dependency of ROOT event from internal event in this library, essential for using these classes safely within ROOT.

Hit classes do not need assigment operators since the working constructors do everything in this case.

\subsection{ROOT splitting and design choices}
\label{sec:root_split}

This note reports the result of studies on ROOT 
class, tree and file handling and the reasons for choices in the design.
It must be understood only in order to modify the internals, 
while users should skip it completely.

ROOT splitting is a big advantage in terms of file I/O efficiency and of graphical tool functionality (e.g. the \code{TBrowser}), 
so the ROOT classes have been designed keeping it as requirement.

\code{BxEvent} class must hold sub classes as objects,
because pointers are not splitted in the \code{TBranch}
and references are not supported by ROOT at all.
This is connected to the fact that ROOT splitting cannot be done for (possibly) polymorphic objects.
The idea of having a polymorphic light/heavy \code{BxEvent} class has therefore been abandoned.
 
A \code{TClonesArray} in the main event class could be included both as an obj and as ptr,
and it would be split correctly in both cases.
However when placed in sub-obj, a \code{TClonesArray} will be splitted only if included as a pointer. 

In turn using pointers to \code{TClonesArray} requires:
\begin{enumerate}
\item The default constructor must construct the \code{TClonesArray} 
or the \code{TTree::Branch()} will segfault (!!!, internally uses the def ctor to see obj size?)
To avoid memory leaks the size is set to 0 in this context.

\item Calling new/delete for \code{BxEvent} class around the \code{TTree::Fill()} in the writer doesn't work.
This is because ROOT caches the pointer to the \code{TClonesArray} at the \code{TBranch::SetAddress()}, 
and these result invalid if \code{BxEvent} is recreated every time. 
It could be solved with a new \code{TBranch::SetAddress()} at every loop, but it is unacceptably slow.
\end{enumerate}

This last point is the reason why a single event is created in the writer and is being rewritten every time with the assigment operators.

Starting with some version of ROOT later then v4.00 the support for STL conbtainers have been improved.
Therefore TClonesArray have been abandoned for std::vector whose use is more straightforward.
They have to be included as objects in order to be split.
In particular the need for point 1 above is removed.
An attempt to create and destroy the object at every event using vectors has not yet been done and is postponed.

After a large correspondance with P.Canal and R.Brun of the ROOT team, 
it has been assessed that nesting lists of whichever kind prevents splitting of the innermost level 
and the ROOT team is not planning to change this due to the cumbersome maintainance it would require.
This is the reason of having hits outside the cluster objects.
