\section{bx\_writer}
\label{sec:writer}

This is a special module regarded as part of the infrastructure of Echidna.

This module handles the ROOT file, the ROOT tree and the ROOT event.

It has the special \code{role=writer}.
It is executed in the main loop after all other modules, i.e. it has the highest priority number.

\subsection{ROOT file} 

A \code{TFile} object is opened in the \code{begin()} method and closed in the \code{end()} method.
The file name is tunable through a user parameter.
The default is "auto", which means that file name will be composed with the std syntax featuring the run number.
If a file with the name requested already exists, by default the program exits with a critic level message, 
unless the proper user parameter to overwrite files is set to true.

Outside the tree, the ROOT file holds also ROOT objects handled by the \code{bx\_root\_barn} (see \ref{sec:barn}).
These are written in the \code{end()} method via \code{bx\_root\_barn::write()} method.


\subsection{ROOT tree}

A \code{TTree} object is created in the \code{begin()} method and written to file in the \code{end()} method.
The ROOT \code{TTree} class is used, since there is no need to develop a dedicated class so far.

A Single \code{TBranch} is created with the \code{BxEvent} class.
Its splitting level\footnote{set to maximum by default.} and buffer size are tunable through user parameters.

The \code{TTree} object is named ``bxtree'', while the \code{TBranch} object is named ``events''.


\subsection{ROOT event}

A single \code{BxEvent} is created in Echidna execution and echidna internal events (\code{bx\_reco\_event}) are copied onto this event before tree is filled.
A technical explanation of this design can be found in \ref{sec:root_event}.

The copy is performed in the \code{doit()} method just before \code{TTree::Fill()} is called. 
It is achieved through the assigment operators defined for \code{BxEvent} class and all its subclasses. 

A helper method \code{m\_parse\_options()} is in charge of reading all user parameters related to which data containers should enabled. 
A struct is filled with these flags and passed to the event constructor.
