\documentclass[a4paper,11pt]{article}
%-------------- CUSTOMIZATION ----------------
\usepackage[vcentering,dvips,left=2.5cm,right=2.5cm,top=2cm,bottom=2cm]{geometry}
\usepackage{setspace}
\parindent=0.5cm
\usepackage[utf8]{inputenc}
\usepackage[T1]{fontenc}
\usepackage{amssymb}
\usepackage{color}
\pagestyle{plain}

\begin{document}
\centerline{\textbf{\LARGE{\color{black}DUTIES OF THE SHIFTERS}}}
\begin{table}[!h]
 \begin{flushright}
   \small{Last update: \today}
 \end{flushright}
\rule{\linewidth}{0.07mm}
\end{table}

\section*{At the beginning of the week}
\begin{enumerate}
\item Make sure you make good contact with the run coordinator (RC) and possibly overlap for a few hours 
with the previous shifter. Ask about the status of the system. Read the paper logbook of the last days.

\item Write your name and cell phone number in the white board outside the Borexino control room, 1$\mathrm{^{st}}$ floor.

\item Become familiar with all the procedures. Note that they change often, so take the time to read them again.

\item Read the whiteboard in the counting room. DAQ parameters and RC contact information are listed there.
\end{enumerate}

\section*{During every shifting day}
\begin{enumerate}
\item The data taking has to be supervised at least 16 hours per day, usually from 08.00 to 24.00. \\
At least one shifter is required to be present in Control Room everyday: \\
~~Morning: 09.00 $\rightarrow$ 12.30 ~~~Afternoon: 14.30 $\rightarrow$ 19.00 \\
For the time no shifter can be underground, tools are provided for remote run monitoring (see DataTakingProcedure.pdf).

\item In case anything unexpected is noticed, please do not hesitate to call the RC; he/she will address the problem 
and eventually call the right expert depending on the problem.\\
Never attempt to operate on the hardware yourself (unless indicated by RC).

\item Write as much info as possible on the paper logbook in clean handwriting.
\begin{itemize}
 \item[a. ] Always put date and hour of each entry.
 \item[b. ] Always write your name when you are starting to write after someone else.
\end{itemize}

\item We generally take data with automatic run restart and a typical run duration of 6 h. \\
Please make sure that the run restart happened at the time you expected; note anomalies in the logbook and inform the RC.

\item When a run ends, the data processing sequence should start automatically, but you need to make sure this is the case. 
Do not leave runs behind as they are often used to monitor the effects of scintillator operations in real time.

\item Fill and update the ELOG (\textit{www.lngs.infn.it/elog/Validation}). 
This is complementary and not redundant compared to the paper logbook. 

\item When the Echidna production is done, start the Validation Procedure. 
This is not a mere procedural execution and requires your skills as a physicist. Follow the scheme and consult with RC if in doubt.

\item In case of flashers/tripping PMTs or high Rate FE boards, call the RC and eventually agree with him/her the application 
of the proper procedure. \\
Never perform any operation without informing the Run Coordinator.

\item Once per day check the Hall C Radon and Temperature and note it in the ELOG.

\item For every run with more than 650,000 events, update the Broken Channels Monitor (see DataHandlingProcedure.pdf).

\item Once a day, create plots for the bad channels (see DataHandlingProcedure.pdf).
\end{enumerate}

\section*{After weekly electronic maintenance}
After weekly electronic maintenance performed by Massino O., George K. and/or Giuseppe B. do an electronics calibration run (see CalibrationProcedure.pdf).
If such operation has not occurred in the last week, inform the RC and discuss when to perform the next electronics calibration.

\section*{At the end of your shift}
\begin{enumerate}
\item Check that all files that you uploaded have been deleted as well as anything you created.
\item Clean the rawdata disk (instructions in DataHandlingProcedure.pdf)
\item Note on the whiteboard anything that might be useful to following shifters.
\item Run the {\small{\texttt{Shift\_analyzer.pl}}}:
\begin{itemize}
 \item[a. ] Become production on bxmaster.
 \item[b. ] {\small{\texttt{cd ~/Echidna/tools}}}
 \item[c. ] {\small{\texttt{./Shift\_analyzer.pl}}}
 \item[d. ] Follow the on-screen instructions and address its output accordingly.
\end{itemize}

\item Write the shift report:
\begin{itemize}
 \item[a. ] Use the provided template (\textit{http://bxweb.lngs.infn.it/docs}).
 \item[b. ] Rename it as sr\_YYYY\_MM\_DD\_NS.txt (where the date is the day your shift started and NS are your initials).
 \item[c. ] Send it by email to {\small{\texttt{bxruncoord@lngs.infn.it}}}.
 \item[d. ] Save the file also as html and copy it to: \\
 {\small{\texttt{/home/daqman/common\_areas/borexwww/docs/ShiftReports/20XX/}}} \\
 If you have permissions problems to do this, contact RC.
 \item[e. ] $\underline{\mathrm{Do~not~send~it}}$ to any mailing list.
\end{itemize}

\item Leave the counting room in a tidy state:
\begin{itemize}
 \item[-  ] Put tools and other materials you used back to their proper locations.
 \item[-  ] Trash your garbage, including water bottles.
 \item[-  ] Trash your notes and any paper you have on the tables in the paper trash bin.
 \item[-  ] Remove from the fridge any food that belongs to you.
\end{itemize} 
\end{enumerate} 
\end{document}
